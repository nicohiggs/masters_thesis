In professional sports, home teams tend to win more on average than visiting teams \cite{Schwartz1977} \cite{Courneya1992} \cite{Nevill1999}. This phenomenon has been widely studied across several fields including psychology \cite{Agnew1994} \cite{Unkelbach2010}, economics \cite{Forrest2005} \cite{Dohmen2016}, and statistics \cite{Buraimo2010} \cite{Lopez2018} among others \cite{Benz2020}. While home advantage is now a widely accepted phenomenon, the magnitude of the advantage and its cause are not as clearly understood or widely accepted as its existence. Part of the difficulty in analyzing the specifics of home advantage is due to the lack of controlled experiments, because nearly every professional game is played in one of the team's home stadium in their home city. While there have existed some show matches at neutral sites, their relative sample sizes are too small from which to draw any reasonable conclusions. For example, the National Football League only plays about 4-5 neutral site games out of a total 256 games each regular season.

The return to play of professional sports during the COVID-19 pandemic presents a unique opportunity to analyze teams playing in situations where home advantage may genuinely no longer apply. The leagues have restricted travel and fan attendance or even created a bubble where only one or two stadiums are used and only the players and necessary staff are present for the games. We consider this restricted return to play as a control group where travel, home stadium familiarity, and home crowd have been controlled (i.e. removed) for enough games to provide a reasonable sample to analyze. There has been considerable academic work analyzing the effect of COVID-19 restrictions on home advantage in European football \cite{Benz2020}. However, comparatively there has been a lack of work analyzing the effect in the North American professional sports leagues. In fact, to the authors knowledge there has only been one work focused on home advantage during COVID-19 across the big four North American professional leagues; and it only investigated the NBA \cite{McHill2020}. In this work, we aim to fill this gap by inferring the effect of and changes in home advantage prior to and during COVID-19 in the big four North American leagues: the National Hockey League (NHL), the National Basketball Association (NBA), Major League Baseball (MLB), and the National Football League (NFL).

Previous works analyzing home advantage tend to pool the results of all teams within a league into one overarching group statistic to analyze, such as overall-league-home-win-percentage. However, more recent work tries to better account for differences amongst teams within a league, such as differences in teams offensive and defensive strengths, by utilizing multiple regression models that account for the effect of other variables while inferring the effect of home advantage. The majority of modern work utilizing more sophisticated regression modeling has been applied to sports outside of the four North American professional leagues considered in this thesis. We fill this gap by developing a Bayesian multilevel regression model and show how the model fits the datasets, of the four professional leagues considered, better than simple averaging and traditional regression modeling.

Professional sports leagues adopted different restrictions in response to the COVID-19 pandemic. The NHL and NBA had the strongest restrictions where they both created a COVID-19 bubble where all games were played at the same consistent location with players quarantined together separate from their families and the outside world. While this proved to be extremely effective in terms of player safety \cite{nhl2020} \cite{usatoday2020} it seems likely that it was the most extreme in terms of its effect on players performance and psychology. In contrast, teams in the MLB and NFL still traveled to their opponents home stadiums. These leagues restricted fan attendance and media access, with some NFL stadiums allowing small amounts of fans to attend. Thus, all leagues lacked a potential home crowd effect, but only the NHL and NBA restrictions removed the additional factors of travel and home city familiarity. Because the restrictions for the NHL and NBA were more strict than those of the MLB and NFL, analyzing all four leagues brings the potential to see similarities and differences across leagues as well as within each individual league. Thus similarities in NHL and NBA as compared to similarities and differences with the MLB and NFL can potentially shed light on the differing effects contributing to home advantage, in particular the differences in the effects of home crowds, familiarity with home cities, and travel. This is noteworthy because of the implications in relation to previous work investigating the causes of home advantage  \cite{Unkelbach2010} \cite{Buraimo2010} \cite{Courneya1992} \cite{Carron2005} \cite{McHill2020} \cite{Garicano2005} \cite{Moskowitz2012}. In McHill \& Chinoy \cite{McHill2020}, the authors argue that home advantage in the NBA's COVID-19 bubble arose from either circadian disruption or the general effect of travel. Our work builds upon such previous works by considering the NBA's COVID-19 bubble and its effects on home advantage while also comparing and contrasting to other similar COVID-19 bubbles in the NHL and different COVID-19 restrictions seen in the MLB and NFL.

\section{Contribution}

We adopt a Bayesian framework to develop a Negative Binomial multilevel regression model that adjusts for relative team strengths while inferring home advantage. We choose this approach for two main reasons. First, alternative methods that rely on correlations among raw statistics, such as home win percentage, fail to account for other factors such as relative team strengths. Our regression approach can infer changes in team performance while adjusting for quality of opponents. Second, the Bayesian framework gives more interpretable results and more flexibility in model building than classical regression methods. The Bayesian framework results in distributions for the estimates of each parameter in our model. This allows us to analyze these distributions directly to determine the probability a parameter is greater (less) than a certain value or that it exists in a specific interval, avoiding the confusion that often arises interpreting p-values and confidence intervals.

By examining the resulting home advantage parameter estimates of our model from before and during the COVID-19 pandemic, we can draw conclusions about the existence of the home advantage phenomenon and provide new evidence for its potential causes. We hypothesize that home advantage is a real phenomenon, thus we expect its parameter estimate to drop during the COVID-19 seasons relative to before the COVID-19 seasons. We are also interested in examining if any differences in relative changes in home advantage exist across the leagues as some leagues had different COVID-19 restrictions which could affect home advantage differently. We also show that point totals in North American professional sports are prone to overdispersion, thus, the Negative Binomial distribution allows for better model fit than the more common Poisson and Normal distributions used in regression analyses. We further show that a multilevel model that pools information for teams offensive and defensive strengths provides better model fit as measured by estimated out-of-sample predictive fit as compared to traditional regression modeling and simple averaging used in many other works of sports modeling and home advantage inference.

The main contributions of this thesis can be summarized as follows:
\begin{enumerate}
	\item First study to provide concrete analysis of home advantage in a controlled setting for the NHL, NBA, MLB, and NFL.
	\item Corroberates results from similar studies analyzing professional European soccer leagues finding a drop in home advantage in some but not all leagues.
	\item Organized a dataset comprised of game results across the NHL, NBA, MLB, and NFL for the years 2016-2020.
	\item Developed a Bayesian multilevel model, provided background on multilevel modeling and evaluation, demonstrated efficacy of model on the datasets used in this thesis.
	\item Demonstrated how North American professional sports are prone to overdispersion in point totals
	\item Proposed a Negative Binomial multilevel regression model to account for overdispersion, evaluated with respect to Poisson and Normal regression models commonly performed in related work.
\end{enumerate}

\section{Thesis Organization}

The rest of the thesis is organized as follows. The Background chapter provides an overview of related work as well as an introduction to Bayesian statistics, multilevel modelling, fitting Bayesian models via Markov Chain Monte Carlo sampling, and evaluating models. The Methods chapter describes in-depth the Bayesian multilevel regression that was developed to infer home advantage, and how the various data experiments for the main contributions of this thesis are set up. The Results chapter presents and discusses the results of the experiments introduced in the Methods chapter. The Conclusions chapter contains a discussion of the results and their implications of the findings and contributions of the thesis.
