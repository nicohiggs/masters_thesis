In this chapter we define the model used throughout the rest of this thesis. We also set up and describe three data experiments that are the main contributions of this thesis.

\section{Multilevel Model} \label{multilevel_model}

- similar to paper
- may need to build this up so as to include all data experiments or a more general version to refer back to

We infer home advantage by fitting a regression model to predict the points scored in each game while adjusting for relative team strengths and home advantage. We adjust for relative team strengths by modelling both an offensive rating and a defensive rating for each team. We argue this better represents real differences between teams and allows the model to better infer if a team performs better or worse when playing at home by measuring its performance relative to its average offensive performance versus its opponents average defensive performance. This section describes in detail the parameters of the model, their interpretation, and how we fit the model.

We aimed to build a parsimonious model to infer home advantage for each league while adjusting for relative team strengths and accounting for uncertainty in the data and parameter estimates. We needed a method that was robust to smaller sample sizes because we only had one COVID-19 adjusted season for each league to compare to and because this sample becomes smaller as you include more parameters which splits the data into smaller groups. We also wanted to be able to quantify the uncertainty in our parameter estimates. To address these concerns we adopt a Bayesian multi-level regression model framework building upon previous work \mbox{\cite{Baio2010} \cite{Glickman1998} \cite{Lopez2018} \cite{Benz2020}} that allows for pooling results across all teams to infer home advantage. The partial-pooling of multi-level regression modelling allows us to separate the effects of individual teams offensive and defensive strengths from their group level means and helps prevent overfitting by adjusting parameter estimates through a process commonly referred to as "shrinkage to the mean" \mbox{\cite{Gelman2014} \cite{Gelman2006} \cite{McElreath2020}}. We argue the pooling of data across each teams results to better handle smaller sample sizes while preventing overfitting, and the ability to quantify the uncertainty in parameter estimates makes Bayesian multi-level regression an ideal choice for this task.

We model the response variable of the number of points scored by each team in each game as Negative Binomial:

where \(y_{ij} = [y_{i1}, y_{i0}]\) is the vector of observed points scored in game \(i\) by the home (\(j=1\)) and away (\(j=0\)) teams and \(\mu_{ij} = [\mu_{i1}, \mu_{i0}]\) are the goal expectations of the home and away teams in game \(i\). The \(\alpha\) parameter allows for the flexibility of fitting to overdispersed data where the variance is much greater than the mean. In our experiments we have found that defining \(\alpha\) as a fraction of \(\mu_{ij}\) led to better sampling and model fit. Thus, we define \(\alpha_{ij} = \mu_{ij} * \lambda\) and then sample \(\lambda\) when fitting the model. We model the logarithm of goal expectation as a linear combination of explanatory variables:

\begin{equation} \label{eq:expected points}
\begin{split}
\text{log}(\mu_{i1}) &= \gamma_{sp} + \beta_{sp} + \omega_{sh[i]} + \delta_{sa[i]} \\
\text{log}(\mu_{i0}) &= \gamma_{sp} + \omega_{sa[i]} + \delta_{sh[i]}
\end{split}
\end{equation}

where \(\gamma_{sp}\) is the intercept term for expected log points in season, with \(s = [0, 1, 2, 3, 4]\) corresponding to the 2016, 2017, 2018, 2019, and 2020 seasons respectively. The subscript \(p\) indicates regular season (\(p=0\)) or playoffs (\(p=1\)). For the results in Figure \ref{fig:ha_pooled}, all previous seasons are combined (\(s=0\)) and compared to the COVID-19 adjusted season (\(s=1\)). Home advantage is represented by \(\beta_{sp}\) with \(s\) and \(p\) the same as the intercept. The offensive and defensive strength of the two teams are represented by \(\omega\) and \(\delta\). The nested indexes \(h[i]\) and \(a[i]\) identify the teams playing at home and away respectively and we use this nested notation to emphasize the multi-level nature of these parameters as they are modelled as exchangeable from a common distribution \cite{McElreath2020} \cite{Gelman2014} \cite{Gelman2006}. This enables pooling of information across games played by all teams in a league and results in mixing of the observable variables \((y_{ij})\) at this higher level which accounts for correlation in home and away points scored in each game \cite{Baio2010}. Note that the offensive and defensive strengths represented this way could potentially lead to problems of identifiability suggested by previous works (cite baio, benz, and karlis) and fully described in (cite mclreath). The issue is that a given difference in relative team strengths can be solved by multiple different team ratings, similar to a system of equations being singular. In line with previous works (baio, benz, karlis), we force the offensive and defensive ratings across all $T$ teams within each league to sum to zero:

\begin{equation} \label{eq:sum to zero}
\sum_{t=1}^{T} \omega_t = 0, \sum_{t=1}^{T} \delta_t = 0
\end{equation}

Not only does this make identifiability a non-issue, but it also improves interpretability of the fitted team ratings as zero represents an average team rating with stronger and weaker ratings being correspondingly above or below zero. Also note that in this formulation defensive ratings are strong (weak) in the negative (positive) direction. This can be seen by considering how a negative defensive rating decreases the expected number of points in \ref{eq:expected points}, thus it represents a strong defensive team. Offensive ratings are the opposite by being strong (weak) in the positive (negative) direction.

In this model formulation we are estimating different home advantage parameters for the regular season and playoffs as well as for each individual season. The primary motivation for this is because the NHL and NBA COVID-19 bubbles essentially only occurred during their playoffs and we therefore want to separate home advantage during the playoffs for a more direct comparison. Modelling in this way also addresses potential questions of whether home advantage changes each year or remains constant. Our results in Figure \mbox{\ref{fig:ha_pooled}} are from estimating one home advantage parameter prior to COVID-19 and one afterwards. We then show the results of modelling home advantage separately for each season and show the results in Figure \mbox{\ref{fig:ha_main}} which reveal some interesting differences as discussed in the Results section.

In (\ref{eq:expected points}) we see that the home team's goal expectation is a linear combination of the home team's offensive strength and the away team's defensive strength as well as a constant home advantage. Conversely, the away team's goal expectation is a linear combination of the away team's offensive strength and the home team's defensive strength with the home advantage parameter noticeably missing. There is no index for league because, although we use the same model consistently across each league, we fit a separate version for each league.

This model formulation results in the intercept representing the logarithm of the overall average of points scored with \(exp(\beta_{sp}), exp(\omega_{sh[i]}),\) and \(exp(\delta_{sa[i]})\) representing multiplicative increases or decreases to the average points scored to determine the expected points scored for an individual game. This can be seen by considering:

\begin{equation}
\begin{split}
\text{log}(\mu_{i1}) &= \gamma_{sp} + \beta_{sp} + \omega_{sh[i]} + \delta_{sa[i]} \\
\mu_{i1} &= \text{exp}(\gamma_{sp} + \beta_{sp} + \omega_{sh[i]} + \delta_{sa[i]}) \\
\mu_{i1} &= \text{exp}(\gamma_{sp})*\text{exp}(\beta_{sp})*\text{exp}(\omega_{sh[i]})*\text{exp}(\delta_{sa[i]})
\end{split}
\end{equation}

For example, a home advantage parameter of \(\beta = 0.25\) would result in multiplying the average points scored by \(\text{exp}(0.25) \approx 1.28,\) which can be interpreted as an increase of about 28\% in expected points scored by the home team in a game between teams with relative offensive and defensive strengths \(\omega_{sh[i]}\) and \(\delta_{sa[i]}\) respectively.

\subsection{Model Fit in PyMC3}

- perhaps a figure of 2 images, one is the 'academic' layout of a model and the other is the pymc3 version showing it as the same but 'reversed order'

The models are fit using PyMC3, an open source probabilistic programming language that allows us to fit Bayesian models with their implementation of a gradient based Hamiltonian Monte Carlo (HMC) No U-Turn Sampler (NUTS) \cite{pymc3}. As in other previous work \cite{Baio2010} \cite{Benz2020}, we use Bayesian modelling and fitting approaches to allow us to incorporate some prior baseline knowledge of parameters as well as better quantifying uncertainty in the interpretation of parameter estimates.

The Bayesian approach means we need to specify suitable prior distributions for all random parameters in the model. The prior distributions for parameters in our model are:

\begin{equation} \label{eq:priors}
\begin{split}
\gamma_{sp} &\sim \mathcal{N}(\theta^*, \sigma^{2*}) \\
\beta_{sp} &\sim \mathcal{N}(0, 1) \\
\lambda &\sim \text{Uniform}(0, 1000) \\
\omega_s &\sim \mathcal{N}(0, \sigma_{s\omega}) \\
\delta_s &\sim \mathcal{N}(0, \sigma_{s\delta}) \\
\sigma_{s\omega} &\sim \text{HalfNormal}(1) \\
\sigma_{s\delta} &\sim \text{HalfNormal}(1)
\end{split}
\end{equation}

where \(\theta^*\) is the logarithm of the average points scored, and \(\sigma^{2*}\) is the logarithm of the variance of points scored, over the regular seasons and playoffs of the league being modelled. We note that we found \(\gamma_{sp}\) fits close to \(\theta^*\) even when using a weakly informative prior, but we keep this formulation as it maintains the spirit of using prior information in Bayesian analysis. We allow \(\lambda\) to potentially be large for instances where there is no overdispersion in the outcome variable because a large \(\lambda\) results in a large \(\alpha_{ij}\) which makes the Negative Binomial distribution become similar to a Poisson distribution.

The model is fit using PyMC3's NUTS sampler using 4 chains of 2000 iterations with 1000 tune steps for a result of 8,000 samples from 12,000 total draws. It is standard practice to check convergence with the \(\hat{R}\) statistic from \cite{Gelman1992} \cite{Brooks1997}.  Each model fit produced \(\hat{R}\) statistics of 1.00 with no divergences \cite{Betancourt2017}.

\section{Data Experiments}

\subsection{Complete pooling, No pooling, and Partial pooling}
- real data showing overfitting and the value of shrinkage
We create and compare three models in order to show the benefits of how multilevel modelling partially pools information across groups to improve model fit while preventing overfitting. The first model is referred to as a \textit{completely pooled} model where the data from all groups is completely pooled into one overall average to be used. The completely pooled model adjusts the model described in section \nameref{multilevel_model} by modifying equation \ref{eq:expected points} as follows:

\begin{equation} \label{eq:cp_model}
\begin{split}
\text{log}(\mu_{1}) &= \gamma_{sp} + \beta_{sp} \\
\text{log}(\mu_{0}) &= \gamma_{sp}
\end{split}
\end{equation}

This is the simplest regression model where an average number of points for home teams and away teams is calculated and then used for predictions.

The second model is referred to as a \textit{no pooling} model where essentially a separate regression fit is made for each group ignoring the information from other groups. This is a traditional regression model and is defined near identically to the model described in section \nameref{multilevel_model}, with the only difference being that the team strength parameters are not mixed. In practice this means that the priors for the team strength parameters $\omega_s$ and $\delta_s$ described in equation \ref{eq:priors} are changed to $\mathcal{N}(0, 1)$. This prevents information being pooled across groups, hence the name no pooling.

Finally the multilevel model is fit as described in section \nameref{multilevel_model}, which is referred to as a \textit{partially pooled} model for the context of this experiment. The effect of this model is shrinking the team strength estimates from the no pooling modeltowards the overall mean. This effect was described in section ?? and in theory helps to prevent overfitting but worsening the fit to the training dataset in order to improve the out-of-sample fit. This experiment is designed to test how this theory holds up on the sports datasets considered in this thesis.

In the completely pooled model there is essentially one global average used for all groups and in this way the model has high bias and ignores the differences amongst groups. In the no pooling model each group has its own parameter fit, but completely ignores the data from other groups and how they are fit resulting in lower bias and a better fit to the data but at the risk of overfitting. The partially pooled model is a balance between these two extremes allowing for a better model fit than the completely pooled model while better protecting against overfitting than the no pooled model.

To compare these models we randomly split each sports dataset in half to create a train-set and test-set. The train-set is be used to train each model and evaluate training fit, and then the model will also be evaluated on the test-set in order to approximate the fit on unseen data. The same fitting and evaluating procedures are performed for each model on each dataset. The results are shown and discussed in section ??.


While one of the primary advantages of Bayesian models is computing a full distribution rather than only a point estimate, we can generate point estimates by taking the mean of the sample distribution of predictions for each model as their respective point estimate predictions to be compared to the actual data. The models are then compared by their mean squared error (MSE), with the results shown in Table ??. There are two key features that stand out in the results in Table ??. First, both the no pooling model and the partial pooling model have lower MSEs in both training and testing. This indicates that the completely pooled model is underfit and that the no pooling and partial pooling model do improve model fit by including group parameters (team ratings in this specific case). Second, the no pooling model provides the best model fit on the training data, but provides worse model fit than the partial pooling model on the test data. This shows the effect of regularization via shrinkage to the mean that was theoretical discussed earlier in section ??. These results give empirical evidence that multilevel models and their partial pooling provide improved model fit while protecting against overfitting, and is the reason we opted to use a multilevel model as the model of choice for inferring home advantage in this thesis.

\subsection{Negative Binomial Regression}
Since point totals in sports are positive integers, the Poisson distribution is a natural choice for modelling their outcomes. The effectiveness of the Poisson distribution for modelling point totals has been shown in several works analyzing European football data \cite{Karlis2003} \cite{Baio2010} \cite{Benz2020}. One shortcoming of the Poisson distribution is that it only has one parameter and this leads to the strong assumption that the mean is equal to the variance. For low scoring sports like European football and hockey, this is usually a fine assumption. However, this is an invalid assumption for several of the sports we analyze in this paper. Table \ref{tab:loo} reports the dispersion statistic \(\sigma_p\). The dispersion statistic represents how much greater the variance is than the mean while adjusting for sample size and model complexity and is computed as  \(\chi^2/(n-p)\) for each league, where \(\chi^2\) is the Pearson chi-squared statistic of the point totals data, and \(n-p\) are the degrees of freedom with \(n\) representing the sample size of the point totals data and \(p\) representing the number of predictors in our model. The commonly suggested threshold, \(\sigma_p > T\), for determining when a Poisson model is no longer appropriate is around \(1.2 < T < 2\) \cite{Payne2018} \cite{Cameron1990}. Table \ref{tab:loo} shows the NBA, MLB, and NFL having potential overdispersion in their point totals and thus, the Poisson distribution is likely inappropriate and less effective. We instead opt for using the Negative Binomial distribution because it has an extra parameter \(\alpha\) that gives greater flexibility and better model fit to data that is overdispersed while still adequately fitting models without overdispersion.

To establish the efficacy of the Negative Binomial distribution in our model, we fit and compare models using the Poisson and Normal distributions across each league. We fit Poisson and Normal regression models by changing the likelihood of the model in (\ref{eq:likelihood}) to \(y_{ij} | \mu_{ij} \sim \text{Pois}(\mu_{ij})\) for the Poisson regression (and subsequently drop \(\alpha\) from the rest of the model as it is not needed), and \(y_{ij} | \mu_{ij}, \sigma^2 \sim \mathcal{N}(\mu_{ij}, \sigma^2)\) for the Normal regression (and use a weakly informative prior \(\sigma^2 \sim \text{HalfNormal}(50)\)). Otherwise the models are identical and their interpretation remains the same as is discussed in the Methods section.

We evaluate the models across each league by estimating the out-of-sample predictive fit via leave-one-out cross-validation (LOO). Following the work of Vehtari \cite{Vehtari2016} we approximate LOO using Pareto-smoothed importance sampling (PSIS) and report the results in Table \ref{tab:loo}. We note here that we also used the widely-applicable information criterion (WAIC) \cite{Watanabe2010} but found the results to be nearly identical and the conclusions the same.

The results can be seen in figure ?? and table ?? and are discussed in section ??.

\subsection{Inferring Home Advantage}
To make inferences about home advantage prior to and during the COVID-19 pandemic we fit model ?? described in section ??. The fitting of this model results in parameter estimated for home advantage across each of the four leagues analyzed for four seasons prior to and one season during COVID-19. We then analyze the trends and differences across seasons as well as leagues in order to make inferences and draw conclusions about the impact of home advantage in these sports. The results can be seen in figure ?? and are discussed in section ??.