The initially most well known and cited work on home advantage in sports was done in 1977 by Schwartz and Barsky \cite{Schwartz1977} who analysed and found home advantage to exist in professional hockey, basketball, baseball and football. In \cite{Courneya1992} the authors accept home advantage as a real phenomena after reviewing the relevant literature and argue for a framework that focuses on game location, psychological states, behavioural states, and performance outcomes to try to understand the underlying causes of home advantage. Follow up work a decade later by Carron et al. \cite{Carron2005} reviewed the literature and concluded that home advantage was still present in both amateur and professional sports, in both individual and team sports, across genders, and across time. More recent works \cite{Pollard2005a} \cite{Gomez2011} confirm the continued existence of home advantage in the North American professional leagues we are considering in this study: the NHL, NBA, NFL, and MLB. In general, older studies on home advantage tend to use correlation methods of aggregated full season statistics (e.g. combining all teams home wins into one home win percentage to see if it is above 50\%), whereas more recent studies generally build statistical regression models from game level data that adjust for additional factors, such as relative team strengths, and try to infer the effect of the home advantage parameter on the regression model.

There have been several studies analysing home advantage in the context of COVID-19 adjusted seasons; however, nearly all of them have focused exclusively on European Soccer leagues. In \cite{Benz2020} thirteen such works are summarized, of which only two used correlation methods and the other eleven made use of regression analysis to infer the change in home advantage. Benz and Lopez themselves use a bivariate Poisson regression model to infer home advantage, thus making for twelve of the fourteen studies making use of regression analysis. Ten of these studies found a drop in home advantage during the COVID-19 adjusted seasons, with the other four reporting mixed results where home advantage dropped in some leagues but not in others. We are only aware of one academic article looking at home advantage in the COVID-19 adjusted seasons for the NBA \cite{McHill2020} where the authors found presence of home advantage prior to the NBA's bubble and argue for teams travel schedules having the most notable impact. As of this writing there are no academic papers examining home advantage during the COVID-19 adjusted seasons for the NHL, NFL, or MLB.. This paper is a first look at using regression to infer home advantage through team performance while adjusting for quality of opponents instead of only looking at aggregated statistics such as win percentage.

There is a growing body of work in sports analytics that turns to building statistical models to measure relative team strengths while accurately predicting game outcomes. These works have their roots found in Bradley-Terry models \cite{Bradley1952} and Bayesian state-space models \cite{Glickman1998}. Further advancements and examples from the NHL, NBA, NFL, and MLB are comprehensively summarized in \cite{Lopez2018} and follow a form similar to the model in \cite{Baio2010} as Bayesian methods generally offer more flexibility to be able to extend and customize these models and are generally more stable when fitting the models to data \cite{GlickmanText2017} while better capturing the uncertainty in estimating parameters opposed to classical point estimates and p-values which are increasingly under criticism in modern science \cite{Ioannidis2005} \cite{Begley2015}. While most of this work was developed with a focus on predicting game outcomes and measuring team strengths, they often include a term to adjust for home advantage and as such can be re-purposed to be used to infer home advantage as is done in the majority of works summarized by \cite{Benz2020}. In this paper we aim to take the first attempt to use these methods to infer home advantage during the COVID-19 adjsuted seasons of the NHL, NBA, NFL, and MLB.

In \cite{Lopez2018} the authors show the improved efficacy of the Poisson distribution instead of the more common Normal distribution \cite{GlickmanText2017} for modelling points scored by each team in each game. In \cite{Benz2020} the authors follow the work in \cite{Karlis2003} arguing for the use of a bivariate Poisson distribution that accounts for small correlation between two teams scoring and show its efficacy over ordinary least squares regression in inferring home advantage via simulations. However, as is shown in \cite{Baio2010} there is no need of the bivariate Poisson when working within the Bayesian framework because multilevel (sometimes referred to as hierarchical) models of two conditionally independent Poisson variables mix the observable variables at the upper level which results in correlations already being taken into account. In \cite{Baio2010} the authors argue for more complex methods to limit the shrinkage of their multilevel model as their data was from leagues with a large range of team strengths. We follow \cite{Lopez2018} who showed that the ``big four'' North American Professional leagues are very close in team strength and thus do not reduce the shrinkage from our multilevel model.

The challenge with methods that look at correlations among raw statistics such as home win percentage is that they fail to account for other factors such as relative team strengths. For example, a weaker team may have poor home win percentage because they have a poor overall win percentage. That same team; however, may perform better at home than they do at other stadiums whilst still losing to stronger opponents and vice versa. This discrepancy can be further impacted by imbalanced schedules. In the professional leagues we consider, teams often do not face each each opponent the same number of times and do not face the same strength of opponents at home and away in a perfectly balanced manner. While studies often recognize this discrepancy, they often claim that it is a small effect that can be ignored \cite{Pollard2005a} without showing evidence. We argue that these issues and any debate over how much of an effect they have is most reliably mitigated by accounting for other factors, most notably team strengths, when trying to infer home advantage. Regression analysis methods are most often used for precisely their ability to account for multiple factors when performing inference, and as such we argue that they are most appropriate for our focus of analysing and inferring home advantage.