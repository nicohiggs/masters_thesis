The results of our model when pooling the previous seasons prior to COVID-19 (Figure \mbox{\ref{fig:ha_pooled}}) show a noticeable decrease in home advantage for the NHL, NBA, and NFL with no noticeable change in home advantage for the MLB. However, while the year-over-year estimates (Figure \mbox{\ref{fig:ha_pooled}}) corroborate these findings to be significant for the NHL and MLB, they show the results are potentially weaker for the NBA and NFL. We argue that the results in Figure \mbox{\ref{fig:ha_main}} reveal that home advantage in the NFL was already decreasing leading up to the 2020 season and that the 2020 COVID-19 restricted season had no significant impact on home advantage in the NFL. We further argue that the NBA COVID-19 restricted season may potentially be an outlier similar to the 2017 playoffs. This means we can not be as confident in our conclusions about home advantage decreasing in the NBA as we are with the NHL. We argue the results give evidence that it is likely home advantage decreased in the NBA but we can not be certain with the limited sample we have.

%The results of our model shows strong evidence of home advantage being a real positive phenomenon in the NHL and NBA prior to the COVID-19 bubble seasons and that the COVID-19 bubble negatively impacted home advantage by removing it almost entirely (Figure \ref{fig:ha_results} (a) and (b)). In contrast, the MLB and NFL showed a trend of relatively smaller home advantage parameter estimates in recent years and showed little to no evidence of COVID-19 restrictions having an impact on home advantage in these sports (Figure \ref{fig:ha_results} (c) and (d)).

If we contrast the COVID-19 restrictions in the NHL and NBA to the MLB and NFL, there are two notable differences. First, the NHL and NBA had much stricter COVID-19 bubbles where teams did not travel to each others stadiums, whereas the MLB and NFL did travel to the various stadiums and only restricted fans attending. This suggests that the lack of travel and home city familiarity contributes to home advantage more than a home crowd effect, and therefore results in a greater drop in home advantage in the leagues that had a strict bubble compared to the leagues that allowed travel and play at home stadiums. This agrees with McHill \& Chinoy \cite{McHill2020} and gives further evidence to the cause of home advantage being more attributable to the general effect of travel. The second difference is the relatively small to no home advantage that the model infers for the MLB and NFL relative to the strongly positive home advantage in recent years found in the NHL and NBA. While we can not fully tease out which of these two differences is stronger, this opens up potential for future work as these leagues continue to play through the COVID-19 pandemic. It will be interesting to see if home advantage returns in the NHL and NBA as they shift toward fewer restrictions similar to the MLB and NFL.

The strongest result for a decrease in home advantage due to COVID-19 restrictions was seen in the NHL. We note that this is particularly interesting because the NHL is somewhat unique to the other leagues, because the home team has an extra difference; they get the last change during stoppages of play, meaning they get to decide player matchups. An analysis of this effect has been carried out by Meghan Hall \cite{Hall2020} who concluded that home teams benefit when they get to control matchups and argued that this benefit should not be discounted during the 2020 COVID-19 bubble season. The results of our model, however, seem to indicate that no home advantage existed during the NHL's COVID-19 bubble and suggests the effect of last change in the NHL is potentially not as impactful as previously thought.

We have also shown how using the Negative Binomial distribution as the likelihood function for our regression model outperforms the Poisson distribution for sports with overdispersion in their point totals such as the MLB and NFL, while still performing just as well as the Poisson distribution when there is little to no overdispersion such as in the NHL and NBA. We showed the Negative Binomial distribution also outperforms the Normal distribution across all leagues except for the NFL where both models vastly outperformed the Poisson distribution. We argue this is because the Negative Binomial distribution effectively represents positive integers like the Poisson distribution while having an extra parameter, like the Normal distribution, to account for overdispersion which represents a greater spread in the data due to greater variance.

We have further shown that a multilevel regression model outperforms traditional regression and simple averaging as measured by out-of-sample predictive fit. Our experiments showed that simple averaging (complete-pooling of, or ignoring, relative team strengths) underfits the data, and that traditional regression is poised to overfit the data. Our multilevel model provides the best tradeoff in providing a better fit to the datasets while preventing overfitting. Combined with Negative Binomial regression, we have introduced a multilevel regression model that fits the data better than previous works and therefore gives better opportunity to infer home advantage.

Our Bayesian regression model has three key advantages over traditional methods for inferring home advantage. First, methods that rely on correlations among raw statistics fail to account for factors such as relative team strengths. For example, a weaker team may have a poor home win percentage because they have a poor overall win percentage. That same team; however, may perform better at home than they do at other stadiums whilst still losing to stronger opponents and vice versa. This discrepancy can be further impacted by imbalanced schedules where teams do not face the same opponents as each other in a perfectly balanced manner. While some studies recognize this discrepancy, they often claim that it is a small effect that can be ignored \cite{Pollard2005a} without showing evidence. We argue that while these claims may hold up for analyses spanning decades they are not appropriate for the short COVID-19 restricted seasons we are considering. Furthermore, these issues and any debate over how much of an effect they have is most reliably mitigated by adjusting for varying team strengths when trying to infer home advantage. Regression analysis methods are primarily used for their ability to account for multiple factors when performing inference, and as such they are most appropriate for our focus of analyzing home advantage. Second, the Bayesian framework gives more interpretable results and more flexibility in model building than classical regression methods. This can be seen in the results of the Bayesian framework being distributions for the estimates of each parameter in our model. In this way the implied probability and corresponding uncertainty of parameter estimates are still rigorously defined while being directly measurable and more intuitive to understand than traditional Frequentist methods of confidence intervals and p-values. Third, with advancements in computational Bayesian statistics, such as Probabilistic Programming languages \cite{pymc3} and Hamiltonian Monte Carlo (HMC) \cite{Betancourt2017}, we are able to easily define and compute flexible and complex models using various likelihood functions with ease instead of being limited to traditional methods like Normal and Poisson regressions more traditionally used in sports modelling \cite{Lopez2018} \cite{Glickman1998} \cite{Karlis2003} \cite{Baio2010} \cite{Benz2020}.

While our model has produced some interesting results, it is worth discussing some of its limitations and areas for future work and improvement. The most notable limitation is that the COVID-19 lockout and restricted seasons are unprecedented and come with additional caveats such as protocols for testing, impact of positive tests, reduced practices, and players being away from their families, that extrapolating all results to home advantage or fan impact alone does not address all the possible factors influencing player and team performance. The model also does not account for travel or rest before games as a potential confounding factor for home advantage. This was ignored primarily due to it being irrelevant for the NHL and NBA COVID-19 bubbles, but for the less restrictive MLB and NFL seasons as well as future COVID-19 restricted seasons this could be a potential factor worth exploring. The model could benefit by including group level factors when estimating the offensive and defensive strengths of teams. The multilevel structure of the Bayesian framework we have adopted naturally allows for such inclusions \cite{Gelman2006} \cite{Gelman2014} \cite{McElreath2020}. For example, we hypothesize that advanced analytics metrics such as expected goals (xG) and corsi in hockey, regularized adjust plus-minus (RAPM) in basketball, hitter splits and park factors in baseball, yards gained/allowed above/below expected in football, could all be leveraged to improve team strength estimates. This could also include personnel differences such as the effect of star players being injured, back-up goalies starting, or starting pitchers being included in the estimates of a teams relative strength for a given game. These inclusions are beyond the scope of this work as these analytics and personnel changes and their effect differ greatly across different sports. In future work, we hope to focus on an individual sport and include such factors, using the current model as a baseline to compare against. Our model is also limited by focusing on only point totals to infer home advantage, while some previous works also analyze differences in penalties to assess a home advantage in the officiating of games \cite{Benz2020} \cite{Unkelbach2010} \cite{Buraimo2010} \cite{Dohmen2016}. This was excluded from this work because of how much penalties and their effect differ across the various sports we considered, but is something we hope to explore in the future when analyzing a single sport in more depth.